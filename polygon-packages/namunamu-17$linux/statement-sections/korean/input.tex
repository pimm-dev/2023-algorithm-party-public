첫째 줄에 최초 정점의 수 N개, 쿼리의 수 M이 주어진다. 
최초의 정점에는 1~N 사이의 번호가 중복되지 않게 붙어있다. (2≤ N , M ≤ 100,000)

둘째 줄에 N개의 정점이 각각 몇 번 정점에서 나온 가지에 있는 것인지 주어진다.
1번은 뿌리이므로 -1이 주어진다.
모든 정점은 최종적으로 뿌리와 한 그래프에 속하지만, 입력 도중에는 속하지 않을 수 있다.

셋째 줄에 N개의 정점이 가지고 있는 열매의 무게가 주어진다. 0이라면 열매가 없는 것이고 뿌리는 열매를 가지지 않는다.

이후 넷째 줄부터 M개의 줄에 걸쳐 쿼리가 주어진다.
