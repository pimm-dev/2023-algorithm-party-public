근성은 나무에 관심이 많다.  
비록 지금은 코딩을 하고 있지만, 그렇다고 나무에 대한 애정이 식은 것은 아니다.
어느 날 이진 트리를 가지고 놀던 근성은 이진 트리는 나무임에도 열매가 안 열린다는 사실을 깨닫고 큰 충격에 빠졌다. 
근성은 나무는 열매가 반드시 열려야 한다 생각하는 나무열매..(중략) 론을 밀고 있었기에 나무 열매가 열리는 트리 그래프를 만들었고 이에 "나무나무"라 이름 지었다.

나무나무의 특징은 다음과 같다.
\begin{enumerate}
\item 이 트리의 1번 정점은 뿌리를 의미한다. 이 트리는 뿌리로부터 위로 뻗어 나간다.
\item 1번 정점을 제외한 정점은 가지가 갈라지는 지점을 의미한다. 정점에는 가지가 연결될 수 있다. 단, 다른 가지에서 갈라진 가지가 정점에서 합쳐지지는 않는다.
\item 간선은 가지를 의미한다. 가지 양 끝에는 반드시 정점이 존재한다.
\item 정점에는 최대 1개의 열매가 열릴 수 있다.
\end{enumerate}

트리를 만든 후 무엇을 할 수 있을까 고민하던 중 아래와 같은 두 가지를 생각해 냈다!
\begin{itemize}
  \item \bf{1 i j w  (접목)} : 임의의 정점 i에 가지를 붙인다.
    \begin{itemize}
      \item 정점 i는 뿌리와 한 그래프에 속하는 정점이다.
      \item 가지 끝에는 정점이 항상 존재하기에 j번으로 번호를 붙인 정점이 가지 반대쪽에 같이 붙는다.
      \item 정점 j에는 w 무게의 열매가 달린다. 열매가 없다면 0이 주어진다.
      \item (1≤ i, j ≤ N + M; 0 ≤ w ≤ 500)
    \end{itemize}
  \item \bf{2 i (수확)} : 정점 i 위로 달린 열매를 모두 떨어트리려면 몇의 힘으로 흔들어야 할지 출력한다.
    \begin{itemize}
      \item 임의의 정점 i를 흔들면 해당 정점 위로 연결된 가지, 정점들이 모두 흔들린다.  
      \item a 무게를 가지는 열매는 a만큼의 힘으로 흔들어야 떨어진다.
      \item 흔들리는 열매가 여러 개면 그만큼 힘이 분산되기에 a 무게와 b 무게의 열매가 있다면 a + b의 힘으로 흔들어야 둘 다 떨어진다.
      \item (1≤ i ≤ N + M )
    \end{itemize}
\end{itemize}


열매는 떨어트려도 나무나무의 특수한 힘으로 다음 쿼리 이전에 자라난다.
쿼리에 주어지는 수는 모두 정수이고, 올바른 입력임을 보장한다. 또한 수확쿼리는 1회 이상 주어진다.


그런데 근성은 이 쿼리를 만들다 갑자기 동아리방에 가야 한다며 도망쳤다.
여러분이 대신 풀어주자
