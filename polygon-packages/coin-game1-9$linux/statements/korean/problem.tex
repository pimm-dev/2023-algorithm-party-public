\begin{problem}{동전 탑 게임 1 }{standard input}{standard output}{1 second}{256 megabytes}

동전 탑 게임은 두 명의 플레이어가 각각 $A, B$개의 동전으로 쌓인 두 개의 탑으로 하는 게임이다. 두 플레이어는 서로 턴을 번갈아 가면서 게임을 진행하게 되고 각각의 턴에는 비어있지 않은 동전 탑 하나를 골라 한 개 이상의 동전을 위에서부터 가져가야 한다. 더 이상 가져갈 동전이 없으면 게임은 끝나고 더 많은 동전을 가져간 플레이어가 승리한다. 만약 서로 같은 양의 동전을 가져갔다면 후공이 승리한다.

민규는 이 게임의 허점을 발견했다! 바로 선공이 너무 간단하게 이길 수 있다는 점이다. 선공이 처음에 동전이 더 많이 쌓인 탑을 골라 동전을 모두 가져가게 된다면 후공은 그 즉시 패배하게 된다. 그래서 민규는 마지막에 동전을 가져가는 사람에게는 추가로 $K$개의 동전을 주는 것으로 위와 같은 사태를 방지하고자 했다.

민규는 개선한 게임을 가지고 영도와 게임을 하려 한다. 영도는 게임을 보자마자 필승법을 눈치채고 재빠르게 선후공 선택권을 가져갔다. 두 플레이어가 최적의 방법으로 게임을 진행한다고 가정했을 때 영도가 이기려면 선공과 후공 중 어느 것을 선택해야 할까.

\InputFile
첫째 줄에 $A,B,K$($1 \le A, B \le 1\,000$) ($0 \le K \le 2\,000$)가 주어진다.

\OutputFile
선공을 선택해야 한다면 First, 후공을 선택해야 한다면 Second를 출력한다.

\Examples

\begin{example}
\exmpfile{example.01}{example.01.a}%
\exmpfile{example.02}{example.02.a}%
\end{example}

\end{problem}

