\begin{problem}{PIMM 파티}{standard input}{standard output}{2 seconds}{1024 megabytes}

PIMM 회장인 민규는 여름방학을 맞이해 파티를 열기로 했다! 회장 역할에 진심인 민규는 \bf{모두가 행복한 파티}를 만들기로 한다.$$$$

수석 인간관계 연구원인 종현이의 말에 따르면, PIMM 회원들은 근성이와 같은 \bf{인싸(I)}와, 윤수와 같은 \bf{아싸(A)}로 분류할 수 있다. \bf{인싸}들은 다른 사람과 말하는 것을 좋아하며, \bf{자신 주변 상하좌우 4칸 안에 최소 1명 이상의 사람이 있으면 행복해한다.} \bf{아싸}들은 다른 사람의 쓸데없는 말을 계속 듣는 것을 싫어하며, \bf{자신 주변 상하좌우 4칸 안에 인싸가 없으면 행복해한다.}

민규는 \bf{행복한 파티}를 위해, $N$×$M$의 직사각형 형태로 이루어진 파티룸을 빌렸다. 민규는 정환이에게서 인싸 $X$명과, 아싸 $Y$명이 파티에 참여한다고 전달받았다.

영도는 민규를 대신하여, 파티룸에 인원들의 자리를 배정하는 역할을 맡게 되었다. 영도를 도와서 \bf{아싸들이 먼저 배치된 파티룸}의 정보가 주어졌을때, \bf{모두가 행복한 파티}를 만드는 경우의 수를 구해보자.

\InputFile
첫 줄에는 $N, M, X, Y$가 한 줄에 주어진다. $(1 \le N,M \le 7, 1 \le X+Y \le N \cdot M)$

다음 N개의 줄에는 \bf{아싸}가 배치된 파티룸의 정보가 주어진다. '*'는 빈 칸이고, 'A'는 \bf{아싸}가 있는 위치이다. \bf{아싸}가 파티룸에 없을 수도 있다.

\OutputFile
\bf{인싸}들을 파티룸의 남은 자리에 배치했을때, \bf{모두가 행복한 파티}를 만들 수 있는 경우의 수를 1,000,000,007로 나눈 나머지를 한 줄에 출력한다. 만약 인원을 어떻게 배치하더라도 모두가 행복할 수 없는 경우, "UNHAPPY"를 출력한다.

\Examples

\begin{example}
\exmpfile{example.01}{example.01.a}%
\exmpfile{example.02}{example.02.a}%
\end{example}

\end{problem}

