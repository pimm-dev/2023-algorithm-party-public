PIMM 회장인 민규는 여름방학을 맞이해 파티를 열기로 했다! 회장 역할에 진심인 민규는 \bf{모두가 행복한 파티}를 만들기로 한다.$$$$

수석 인간관계 연구원인 종현이의 말에 따르면, PIMM 회원들은 근성이와 같은 \bf{인싸(I)}와, 윤수와 같은 \bf{아싸(A)}로 분류할 수 있다. \bf{인싸}들은 다른 사람과 말하는 것을 좋아하며, \bf{자신 주변 상하좌우 4칸 안에 최소 1명 이상의 사람이 있으면 행복해한다.} \bf{아싸}들은 다른 사람의 쓸데없는 말을 계속 듣는 것을 싫어하며, \bf{자신 주변 상하좌우 4칸 안에 인싸가 없으면 행복해한다.}

민규는 \bf{행복한 파티}를 위해, $N$×$M$의 직사각형 형태로 이루어진 파티룸을 빌렸다. 민규는 정환이에게서 인싸 $X$명과, 아싸 $Y$명이 파티에 참여한다고 전달받았다.

영도는 민규를 대신하여, 파티룸에 인원들의 자리를 배정하는 역할을 맡게 되었다. 영도를 도와서 \bf{아싸들이 먼저 배치된 파티룸}의 정보가 주어졌을때, \bf{모두가 행복한 파티}를 만드는 경우의 수를 구해보자.
