\begin{problem}{인기투표}{standard input}{standard output}{1 second}{256 megabytes}

종현이는 군 생활을 기록할 겸 증명사진을 찍어보았다. 그런데 왠걸! 사진관으로부터 받은 증명사진 속엔 대학생 상병 박종현이 아니라 세상 고초 다 겪은 직업 군인 중사 박종현이 찍혀있었다. 사진에 불만이 생긴 종현이는 생활관 동기들에게 하소연을 했지만, 동기들은 종현이를 놀리기에 바빴다. 억울한 종현이는 SNS에 이 사진이 정말로 자신과 닮았는지 공개 투표를 부치기로 했다. 투표는 백분율만 공개된다.

근성이는 공개 투표가 인싸들의 전유물이라고 생각했기 때문에, 감히 친구가 적은 종현이가 스토리에 자신의 증명사진과 공개 투표를 올린 것을 용서하지 못한다. 하지만 근성이의 생각과 달리 종현이가 새로 사귄 친구가 많을 수도 있으므로, 투표 상황을 지켜보고 종현이의 인기를 추정해보기로 했다:

백분율로 나타나는 투표 결과로부터 전체 투표수를 추측할 수 있으므로, 과거 어느 시점에서의 투표 결과와 현재의 투표 결과를 바탕으로 현재의 총 투표수를 추정한다.

근성이는 종현이가 친구가 많을 것이라고 생각하지 않기 때문에, 가능한 총 투표수(총 투표 참여자 수) 결과가 여러개라면 근성이는 가장 낮은 값을 결과로 받아들일 것이다.

투표는 한번 하면 번복할 수 없다.

\InputFile
첫째 줄에 투표 항목의 개수 $N$ ($1\leq N\leq100$) 과 투표 결과가 표현되는 소수점 자리수 $P$ ($0\leq P\leq6$)가 주어진다.

둘째 줄에 이전 시점, 백분율로 표현되는 각 항목의 투표 결과에 $10^P$를 곱한 값이 $N$개 주어진다.

둘째 줄에 현재 시점, 백분율로 표현되는 각 항목의 투표 결과에 $10^P$를 곱한 값이 $N$개 주어진다.

둘째 줄과 셋째 줄에 주어지는 각 값은 반올림이 이루어지지 않은 값이다. (각 시점에서 주어지는 수의 합은 10의 거듭제곱임이 보장된다.)

\OutputFile
근성이가 추정하는 이전 시점의 총 투표수와 현재 시점의 총 투표수를 공백을 간격으로 순서대로 출력한다.

\Examples

\begin{example}
\exmpfile{example.01}{example.01.a}%
\exmpfile{example.02}{example.02.a}%
\end{example}

\end{problem}

